\section{Реализация дискретно-событийной библиотеки Simlibrary для моделирования СХД}

\subsection{Описание Environment имитации}

Имитация системы происходит при помощи объекта типа Environment, который полностью отражает текущее положение системы. Данный объект обладает следующими полями:

\textbf{currentTime}
Текущее время системы типа $float64$. Изменяется дискретными шагами.

\textbf{workers}
Словарь типа $map[uint64]*Process$, где в качестве ключа выступает идентификатор PID объекта, а в качестве значение указатель $Worker$.

\textbf{routesMap}  
Словарь типа  $map[Route]*Link$. Содержит значения обо всех путях, возможной в данной конфигурации компьютерной сети. $Route$ -- структура, имеющая в качестве своих полей $start$ и $finish$ -- указатели на хост, которые являются началом и концом пути, соответственно. Значение $*Link$ -- является указателем на сеть, по которой будет проходить передача пакетов в обе стороны.


\textbf{queue}
Поле типа $eventQueue$. Является глобавльным хранилищем всех событий во время имитации сложных сложных систем.

\textbf{mutex}         
Поле типа $sync.Mutex$. Примитив синхронизации нужный для того, чтобы обеспечивать консистентность доступа к данным, таким как очередь событий $queue$. 

\textbf{shouldStop}    
Поле типа $bool$. Во время прогона симуляции является равным 1. После того, как все события в имитации заканчиваются, либо при наличии специального события, выставляется в отрицательное значение и прогон прекращается.

\textbf{hostsMap}      
Словарь типа $map[string]HostInterface$. Содержит информацию обо всех хостах, которые имеются в симуляции. В качестве ключа словаря -- имя хоста, в качестве значения интерфейс типа HostInterface, который обобщает такие типы как $host$, $NetworkSwitch$ и $IOBalancer$.

\textbf{vesninServers}
Поле типа $[]*Host$. Является списком, который содержит информацию о рабочих дисковых контроллерах, поддреживающих отношения с клиентом, представленных в виде указателя на $*Host$.

\textbf{allVesninServers}
Поле типа $[]*Host$. Поле типа $[]*Host$. Является списком, который содержит информацию о обо всех (рабочих и нерабочих) дисковых контроллерах, поддреживающих отношения с клиентом, представленных в виде указателя на $*Host$.

\textbf{storagesMap}   
Словарь типа $map[string]*Storage$. В данном контейнере хранится информация обо всех дисках, примонтированных к системе хранения данных. В качестве ключа словаря используется идентификатор конечного хранилища данных, а качестве значения -- указатель на дисковое хранилище.

\textbf{linksMap}      
Словарь типа $map[string]*Link$. Контейнер, содержащий информацию обо всех сетях, представленных в данной симуляции.  В качестве ключа словаря используется идентификатор сети, а качестве значения -- сеть, по которой будет идти передача данных.

\textbf{FunctionsMap} 
Словарь типа $map[string]func(*Process, []string)$. Данный контейнер хранит информацию обо всех фукнциях, которые будует запущены в качестве горутин в начальный момент времени (время запуска симуляции). Функции должны быть объявлены в файле deployment.xml.  В качестве ключа словаря используется идентификатор функции, представленный в строковом виде, а качестве значения -- указатель на функцию.

\textbf{daemonList}  
Поле типа $[]*Process$. Является списком, который содержит информацию о горутинах, которые во время симуляции СХД, являются представлениями Unix-демонов и самостоятельно должны завершить своё исполнение.

\textbf{pid} 
Поле типа $ProcessID$. Указатель на функцию, которая исполняется в текущий момент времени.

\textbf{waitWorkerAmount} 
Поле типа $uint64$. Количество горутин, запущенных в текущий момент времени, завершения которых нужно ожидать для того, чтобы наполнить очередь актуальными текущими событиями. 

\textbf{stepEnd}        
Поле типа $chan interface{}$. Является средством коммукации горутин с главной ($master$) горутиной. В данный канал связи горутины, которые исполняются в текущий момент времени, сигнализируют о своём завершении.

\textbf{nextWorkers}
Поле типа $[]*Process$. Является списком, который содержит информацию о горутинах, которые должны буть запущены на следующем шаге работы имитации системы со статусом $OK$.

\textbf{timeOutWorkers}   
Поле типа $[]*Process$. Является списком, который содержит информацию о горутинах, которые должны буть запущены на следующем шаге работы имитации системы со статусом $TIMEOUT$.

\textbf{anomalyWorkers}   
Поле типа $[]*Process$. Является списком, который содержит информацию о горутинах, которые должны буть запущены на следующем шаге работы имитации системы со статусом $FAIL$.

\textbf{logsMap}  
Словарь типа $map[string]float64$. Данный контейнер хранит информацию, которая впоследствии будет выведена в виде логов системы.  В качестве ключа словаря используется идентификатор наблюдаемого значения, а качестве значения -- числовая характеристика данной величины.

\textbf{unitsMap} 
Словарь типа $map[string]float64$. Данный контейнер хранит информацию о единицах системы измерений, принятых в данной симуляции. В качестве ключа словаря используется идентификатор единицы измерения, а качестве значения -- численная характерстика относительно эталона.

\textbf{backupRoutesMap} 
Словарь типа  $map[Route]*Link$. Содержит значения обо всех запасных (backup) путях, возможной в данной конфигурации компьютерной сети. $Route$ -- структура, имеющая в качестве своих полей $start$ и $finish$ -- указатели на хост, которые являются началом и концом пути, соответственно. Значение $*Link$ -- является указателем на сеть, по которой будет проходить передача пакетов в обе стороны.

\textbf{HostLinksMap}
Словарь типа $map[HostInterface][]*Link$. Данный контейнер хранит информацию о сетях, к которым имеет доступ каждый хост. В качестве ключа словаря используется идентификатор хоста, а качестве значения -- список, состоящий из указателей на сеть, принадлежащих даннному хосту.

\textbf{LinkBackupsMap}  
Словарь типа $map[*Link]*Link$.  В качестве ключа словаря используется идентификатор конечного хранилища данных, а качестве значения -- указатель на дисковое хранилище.

\subsubsection{Функции Environment}
\textbf{func NewEnvironment() *Environment}

Входные аргументы: отсутсвуют.

Выходное значение: переменнная типа *Environment.

Описание функции:Создаёт и инициализует необходимые поля для функцирования симуляции, такие как:
\begin{itemize}
\item 		queue           
\item 		workers        
\item 		SendEventsNameMap 
\item 		ReceiveEventsNameMap
\item 		ReceiverSendersMap
\item 		stepEnd
\item 		logsMap
\item 		HostLinksMap
\item 		LinkBackupsMap
\end{itemize}

\textbf{func createUnits()}

Входные аргументы: отсутствуют.

Выходное значение: отсутствует.

Описание функции: Инциализирует единицы измерения необходимые при симуляции системы.
\begin{table}[]
\centering
\caption{My caption}
\label{my-label}
\begin{tabularx}{\textwidth}{|X|X|}
TB   & $1000^4$ byte         \\
GB   & $1000^3$              \\
MB   & $1000^2$              \\
KB   & $1000$                \\
B    & $1$                   \\
GBps & $1000^3$ byte per sec \\
MBps & $1000^2$ byte per sec \\
KBps & $1000$ byte per sec   \\
Bps  & $1$ byte per sec      \\
Gf   & $1000^3$ flops        \\
Mf   & $1000^2$ flops        \\
Kf   & $1000$ flops          \\
f    & $1$ flops            
\end{tabularx}
\end{table}

\textbf{func (env *Environment) stopSimulation(EventInterface)}

Входные аргументы: указатель на объект типа *Environment.

Выходное значение: отсутствует.

Описание: Даннная функция останавливает исполнение программы путем выставления флага shouldStop в положительное значение.

\textbf{func (env *Environment) updateQueue(deltaTime float64) }

Входные аргументы: указатель на объект типа *Environment.

Выходное значение: отсутствует.

Описание: Даннная функция обновляет очередь событий за время $deltaTime$. 

\textbf{func (env *Environment) CreateTransferEvents()}

Входные аргументы: указатель на объект типа *Environment.

Выходное значение: отсутствует.

Описание: Даннная функция создаёт события, которые имитируют передачу данных от одного хоста к другому.

\textbf{func (env *Environment) Step() EventInterface }

Входные аргументы: указатель на объект типа *Environment.

Выходное значение: текущее событие симуляции.

Описание: Даннная функция осуществляет шаг симуляции, который состоит из следующих шагов. 
\begin{enumerate}
\item Cоздать события, которые имитируют передачу данных от одного хоста к другому.
\item Проверить является ли этот шаг симуляции последним.
\item Проверить симуляцию на возникновение дедлоков.
\item Получить событие из очереди с минимальным значением времени.
\item Обновить текущее время.
\item Обновить очередь событий за время, прошедшее с времени прошлого события.
\item Обработать коллбэки (callbaks) текущего события.
\item Проверить является ли этот шаг симуляции последним.

\end{enumerate}

\textbf{func (env *Environment) FindNextWorkers(event EventInterface)}

Входные аргументы: указатель на объект типа *Environment, текущее событие EventInterface.

Выходное значение: отсутствует. 

Описание: Даннная функция занимается поиском горутин, которые должны начать исполнение после выполнения текущего шага. Данный список включает в себе также горутины, которые начнут исполнение со статусами $OK$, $FAIL$, $TIMEOUT$.

\textbf{func (env *Environment) SendStartToSignalWorkers()}

Входные аргументы: указатель на объект типа *Environment.

Выходное значение: отсутствует. 

Описание: Даннная функция рассылает сигналы через каналы коммуикации горутинам, которые должны начать исполнение после выполнения текущего шага. Данный список включает в себе также горутины, которые начнут исполнение со статусами $OK$, $FAIL$, $TIMEOUT$.

\textbf{func (env *Environment) WaitWorkers() }

Входные аргументы: указатель на объект типа *Environment.

Выходное значение: отсутствует. 

Описание: Даннная функция дожидается выполнения задач текущими горутинами, которым были посланы сигналы на предыдущем этапе.



\subsection{Описание примитивов, использованных при имитации системы}

\subsubsection{Имитация сети}

Сеть имитируется при помощи структуры $Link$. Она обладает следующими полями (характеристиками). 

\textbf{name}
Идентификатор сети в текстовом представлении типа string. 

\textbf{state}	 float64
Степень соответсвия изначальному ресурсу, либо 1 минус деградация данной сети. Значение типа float64, может принимать значения от 0 до 1, где 0 соответсвует полной деградации сети, а 1 -- "фабричному" состоянию. 

\textbf{route}	 *Route
Указатель на структуру данных Route, которая содержит информацию о хостах, которые соединяет данная сеть. 

\textbf{minEvent}	
Указатель на минимальное событие-пакет *TransferEvent, которое передаётся в текущий момент по сети. 

\textbf{bandwidth}	 
Переменная типа float64. Пропускная способность сети, которая изменяется в байтах в секунду. 

\textbf{lastTimeRequest}
Переменная типа float64. Время последнего обращения к данной сети.

\textbf{mutex}	           
Примитив синхронизации типа sync.Mutex необходимой для корректности парального доступа к полям структуры данной сети.
 
\textbf{counter}
Переменная типа int64. Количество пакетов, которые передаются в текущий момент времени по сети. 


\textbf{func NewLink(bandwidth float64, name string) *Link}

Входные аргументы: bandwidth -- переменная типа float64. Содержит информацию о пропускной способности сети. name -- переменная типа string, имя сети. 

Выходное значение: Указатель созданную структуру, которая инкапсулирует сеть. 

Описание функции: Создаёт указатель созданную структуру, которая инкапсулирует сеть с именем name и пропускной способностью bandwidth и инициализирует необходимые поля сети, такие как:

\begin{itemize}
\item bandwidth 
\item mutex    
\item name   
\item state  
\end{itemize}




\textbf{func (link *Link) Put(e *TransferEvent)}

Входные аргументы: Указатель на структуру *Link, указатель на событие, которое должно передаваться по сети.

Выходное значение: отсутствует.

Описание функции: Данная функция добавляет событие в очередь событий, относящейся к сети link. 


\textbf{func (link *Link) EstimateTimeEnd(e *SendEvent)}

Входные аргументы: Указатель на структуру Link; указатель на событие SendEvent, которое должно передаваться по сети.

Выходное значение: отсутствует

Описание функции: Оценить время окончания  $t_{end}$ передачи события-пакета по данной сети по следующей формуле:

\[ t_{end} = t_0 + \dfrac{S}{\dfrac{B}{n}  \cdot q }  \], 

где $t_0$ -- это текущее время, \(S\) -- размер передаваемого пакета,  \(B\) -- пропускная способность сети, \(n\) -- количество пакетов, которые передаются в текущий момент времени,  \(q\) -- степень деградации сети. 


\textbf{func (env *Environment) FindNextTransferEvent()}

Входные аргументы: Указатель на структуру *Environment.

Выходное значение: Отсутствует. 

Описание функции: Данная функция "составляет" события, которые будут передаваться в текущий момент времени.  


\textbf{func GetRoute(route Route) *Link}

Входные аргументы: route переменна типа Route, содержащая информацию об начальном и конечном хостах.

Выходное значение: Указатель на структуру Link. 

Описание функции: Данная функция по имени route возвращает указатель на структуру Link.


\textbf{Route} обладает следующими полями.
Указатель на начальный	start. Тип  HostInterface

Указатель на конечный finish. Тип HostInterface

\textbf{}

Входные аргументы: отсутствуют.

Выходное значение:

Описание функции:


\subsubsection{Имитация хоста}
\textbf{name}      
Поле типа string. Является идентификатором объекта.  

\textbf{typeId} 
Поле типа string. Содержит информацию о классе устройств, которым принадлежит данный хост. 

\textbf{processes} 
Поле типа []*Process. Содержит информацию в виде списка указателей на Process обо всех текущих процессах, запущенных на данном хосте. 

\textbf{speed}     
Поле типа float64. Скорость работу данного хоста, измеряемая в flops.

\textbf{storage}   
Поле типа *Storage. Содержит указатель на диск, который примонтирован к данному хосту.

\textbf{traffic} 
Поле типа float64. Траффик в байт/с, который проходит через данный хост.

\textbf{logs} 
Поле типа interface\{\}. Текстовове представление логов данного хоста.

\textbf{Функции необходимые для имитации хоста}

\textbf{func (env *Environment) getHostByName(name string) HostInterface}

Входные аргументы: Аргумент name типа string.

Выходное значение: Объект типа HostInterface.

Описание функции: Данная функция по данному имени name возвращает объект типа HostInterface.


\textbf{func (process *Process) GetHost() HostInterface }

Входные аргументы: Указатель на процесс, владеющий в данное время исполнением.

Выходное значение: Объект типа HostInterface

Описание функции: Данная функция возвращает хост HostInterface, на котором в данное время исполняется текуща горутина. 

\textbf{func (host *Host) GetName() string }

Входные аргументы: Указатель на объект Host.

Выходное значение: Идентификатор хоста тип string.

Описание функции: Данная функция возвращает имя текущего хоста.


\textbf{func (host *Host) GetType() string }

Входные аргументы: Указатель на объект Host.

Выходное значение: Тип класса устройств к которым относится данный хост. Текстовое представление.

Описание функции: Данная функция возвращает тип текущего хоста.


\textbf{func (host *Host) GetDevTemp() float64 }

Входные аргументы: Указатель на объект Host.

Выходное значение: Температура данного хоста. Тип float64.

Описание функции: Данная функция возвращает 


\textbf{func (host *Host) GetTraffic() float64}

Входные аргументы: Указатель на объект Host.

Выходное значение:

Описание функции: Данная функция возвращает 


\textbf{func (host *Host) AddTraffic(traffic float64)}

Входные аргументы: Указатель на объект Host.

Выходное значение:

Описание функции: Данная функция возвращает 


\textbf{func (host *Host) GetLoad() int }

Входные аргументы: Указатель на объект Host.

Выходное значение:

Описание функции: Данная функция возвращает 


\textbf{func (host *Host) GetLogs() interface}

Входные аргументы: Указатель на объект Host.

Выходное значение:

Описание функции: Данная функция возвращает 


\textbf{func (host *Host) SetLogs(logs interface)}

Входные аргументы: Указатель на объект Host.

Выходное значение:

Описание функции:


\textbf{func (host *Host) GetStorage() *Storage }

Входные аргументы: Указатель на объект Host.

Выходное значение:

Описание функции: Данная функция возвращает 


\textbf{func GetHostByName(hostName string) HostInterface }

Входные аргументы: Указатель на объект Host.

Выходное значение:

Описание функции: Данная функция возвращает 


\subsection{Устройство очереди}

Наиболее важным элементом при реализации моделирования сложных систем является поддержание консистентности очереди событий. В текущей версии библиотеки она реализована при помощи встроенного в язык программирования интерфейса "container/heap". Данный интерфейс представляет структуру данных под названием дерево, которое обладает свойством, что каждая его узел является минимальным значением в его поддереве. Эта структура данных была выбрана для моделирования, т.к является наиболее распространенной при реализации очереди событий с приоритетом, которым в случае моделирования событийных имитаций является время окончания события. 

Для имплементации данного интерфейса были реализованы следующие функции: 

\begin{table}[]
\centering
\caption{My caption}
\label{my-label}
\begin{tabularx}{\textwidth}{|X|X|X|X|}
\hline
Функция                                      & Входные аргументы                                                          & Выходные аргументы                              & Описание функции                                                                                                                                                                                                         \\ \hline
func (eq eventQueue) Len() int               & -                                                                          & Длина очереди, тип int                          & Данная функция возвращает значение длины очереди.                                                                                                                                                                        \\ \hline
func (eq eventQueue) Less(i, j int) bool     & Индексы элементов очереди. Тип int                                         & Тип bool                                        & Данная функция задаёт правило сравнения и  сравнивает элемент очереди с индексом i c элементом с индексом j. В случае, если первый элемент больше, то возвращается логическое да, в противном случает -- логическое нет. \\ \hline
func (eq eventQueue) Swap(i, j int)          & Индексы элементов очереди. Тип int                                         & -                                               & Данная функция меняет местами элемент очереди с индексом i c элементов очереди с идексом j.                                                                                                                              \\ \hline
func (eq *eventQueue) Push(e interface\{\})  & Значение, которое необходимо добавить в очередь. Тип interface\{\}         & -                                               & Данная функция добавляет новый элемент e в очередь событий.                                                                                                                                                              \\ \hline
func (eq *eventQueue) Pop() interface\{\}    & -                                                                          & Минимальнй элемент в очереди. Тип interface\{\} & Данная функция извлекает минимальный элемент из очереди событий.                                                                                                                                                         \\ \hline
func(eq *eventQueue) Fix(h Interface, i int) & h - элемент в очереди, который нуждается в изменениях. i - новый приоритет & -                                               & Данная функция меняет приоритет у элемента h в очереди событий на приоритет i.                                                                                                                                           \\ \hline
\end{tabularx}
\end{table}



